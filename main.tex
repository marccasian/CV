%%%%%%%%%%%%%%%%%%%%%%%%%%%%%%%%%%%%%%%%%
% Medium Length Professional CV
% LaTeX Template
% Version 2.0 (8/5/13)
%
% This template has been downloaded from:
% http://www.LaTeXTemplates.com
%
% Original author:
% Rishi Shah 
%
% Important note:
% This template requires the resume.cls file to be in the same directory as the
% .tex file. The resume.cls file provides the resume style used for structuring the
% document.
%
%%%%%%%%%%%%%%%%%%%%%%%%%%%%%%%%%%%%%%%%%

%----------------------------------------------------------------------------------------
%	PACKAGES AND OTHER DOCUMENT CONFIGURATIONS
%----------------------------------------------------------------------------------------

\documentclass{resume} % Use the custom resume.cls style
\usepackage{textcomp}
\usepackage[left=0.75in,top=0.6in,right=0.75in,bottom=0.6in]{geometry} % Document margins
\newcommand{\tab}[1]{\hspace{.2667\textwidth}\rlap{#1}}
\newcommand{\itab}[1]{\hspace{0em}\rlap{#1}}
\name{MARC Casian-Nicolae} % Your name
\address{} % Your address
%\address{123 Pleasant Lane \\ City, State 12345} % Your secondary addess (optional)
\address{(+40)747127916 \\ marccasiannicolae@gmail.com \\ https://github.com/marccasian} % Your phone number and email

\begin{document}

%----------------------------------------------------------------------------------------
%	EDUCATION SECTION
%----------------------------------------------------------------------------------------

\begin{rSection}{Education}

{\bf National College "Avram Iancu", Campeni} \hfill {\em 2011-2015} 
\\ Mathematics and Computer Science, GPA-9.93/10
\\{\bf University “Babeș-Bolyai”, Cluj-Napoca} \hfill {\em 2015-2018} 
\\Faculty of Mathematics and Computer Science, Bachelor degree in Computer Science\\ \hfill { Thesis: 10/10; Bachelor Exam: 9/10; GPA: 9.8/10}
\\{\bf University “Babeș-Bolyai”, Cluj-Napoca} \hfill {\em 2018-2020} 
\\Faculty of Mathematics and Computer Science, Master degree in Applied Computer Intelligence
\\{\bf Johannes Kepler University, Linz} \hfill {\em Feb-July 2019} 
\\Exchange student, Master degree in Applied Computer Intelligence
%Minor in Linguistics \smallskip \\
%Member of Eta Kappa Nu \\
%Member of Upsilon Pi Epsilon \\

%--------------------------------------------------------------------------------
%    WORK EXPERIENCE
%----------------------------------------------------------------------------------------------

\end{rSection}

\begin{rSection}{Work experience}

    \begin{rSubsection}{MindCoding, Programming contest- Cluj-Napoca, Romania}{2016 - 2017}{Scientific Committee Member}{3rd \& 4th Edition}
    \end{rSubsection}
    \begin{rSubsection}{LearnHouse- Cluj-Napoca, Romania}{May 2016 - October 2016}{Volunteer teacher}{}
    \end{rSubsection}
    \begin{rSubsection}{Bitdefender - Cluj-Napoca, Romania}{February 2016 - July 2020}{Trainee \hfill Feb 2016 - May 2016}{\\Junior Software Engineer \hfill Jul 2016 - Feb 2018\\Software Engineer \hfill Feb 2018 - Jul 2020\\Scrum master \hfill Jun 2018 - Feb 2019\\Technical Lead \hfill Jan 2020 - Jul 2020}{}
    \end{rSubsection}
    \begin{rSubsection}{Amazon - Madrid, Spain}{August 2020 - Present}{Software Engineer Developer II}{}
    \end{rSubsection}

\end{rSection}
%--------------------------------------------------------------------------------
%    Projects 
%-----------------------------------------------------------------------------------------------
\begin{rSection}{Personal Projects}
{\bf KarySOM, KaryML} - Research project on automated karyotyping problem.\\
{\bf Conference Management System} - Coordinated a team of six students during development of a university project\\
{\bf EpiFriend(PoC)} - developed in a team of 5 members. It is an application that detects an epilepsy crisis using Hitoe t-shirt and send alerts to yours trusted contacts (friends)\\
{\bf Face Mask Detector(PoC)} - developed in a team of 9 members during a 48h coding marathon against COVID-19. It is an AI based approach to detect people that are not wearing facial masks in a public environment
\end{rSection}
%----------------------------------------------------------------------------------------
%	SCIENTIFIC WORK
%----------------------------------------------------------------------------------------

\begin{rSection}{Scientific publications}
\emph{{\bf KarySOM}: An Unsupervised Learning based Approach for Human Karyotyping using Self-Organizing Maps
Casian-Nicolae Marc-Gabriela Czibula - 2018 IEEE 14th International Conference on Intelligent Computer Communication and Processing (ICCP) - 2018}


\emph{{\bf KaryML}: A study towards using unsupervised learning in automating Human Karyotyping.
Student Scientific Communication - 1st place - Best student paper - “Babeș-Bolyai" University - 2020}
\end{rSection}
%----------------------------------------------------------------------------------------
%	SKILLS
%----------------------------------------------------------------------------------------
\newpage
\begin{rSection}{Skills}
    \begin{tabular}{ @{} >{\bfseries}l @{\hspace{6ex}} l }
        Python, MySQL, GIT \ & Experience level: ~4 years software development and personal projects \\
        Linux \ & Experience level: used almost daily \\
        Machine Learning \ & Experience level: personal projects, research projects \\
        Java, PHP, Batch, Shell \ & Experience level: personal projects and educational projects (medium)\\\\
    \end{tabular}
\end{rSection}



%----------------------------------------------------------------------------------------
%	AWARDS
%----------------------------------------------------------------------------------------

\begin{rSection}{Awards}
\begin{tabular}{ @{} >{\bfseries}l @{\hspace{6ex}} l }
Informatics Olympiad \ & County phase: 2nd Place 2014, 1st Place 2015\\ \ & National phase: 34th Place 2015\\
Catalysts Coding Contest \ & 3rd Place 2016 (Cluj-Napoca, Romania)\\ \ & 6th place 2019 (Linz, Austria)\\ \ & 2nd place 2020 (Oradea, Romania)\\
Others \ & Multiple awards and mentions at inter-county algorithmic competitions
\end{tabular}


\end{rSection}

%----------------------------------------------------------------------------------------
%	RELEVANT ACTIVITIES
%----------------------------------------------------------------------------------------
\begin{rSection}{Personal Traits}    
    Highly motivated and eager to learn new things.\\
    Strong motivational and leadership skills.\\
    Ability to work as an individual as well as in group.\\
    I\textquotesingle m the one who brings the good vibe.

\end{rSection}

\begin{rSection}{Volunteer activities} 
 
    Project “\textbf{Live positive life}” and other activities in the student’s school council (high school)\\
    \textbf{Hermes} (Mathematics and Computer Science students association) Oct 2015 - Jan 2017 

\end{rSection}

%----------------------------------------------------------------------------------------
% CONTACT
%----------------------------------------------------------------------------------------

% \begin{rSection}{Contact} 
%  \begin{tabular}{ @{} >{\bfseries}l @{\hspace{6ex}} l }
% Home address \ & Nr.132, Musca, Lupsa, Alba county, postal code: 517424, Romania\\
% Mobile \ & (+40) 747 127 916 \\

%  \end{tabular}
% \end{rSection}

% \newpage

% \begin{rSection}{Extra-Cirrucular} \itemsep -3pt
% \item Co-Organized “ Nirmitee 2017” - a National Symposium of Civil Department of MIT, Pune
% \item Attended a workshop on Autodesk Revit at IIT Bombay in 2014.
% \item Winner of Inter Departmental Football Competition 2015.
% \item Member of the  Rotaract Club Of Pune Pride from 2014 to 2017.
% \item Worked for a start-up company Named OUST as a Regional Marketing Manager
% %\item Trained and disciplined in National Cadet Corps (NCC), IIT Kanpur for a year.
%  %\item  Participated in Vijyoshi Camp 2012 organized at Indian Institute of Science, Bangalore.
%  %\item Won 2nd position in Kho-Kho in Intramurals conducted by Physical Education Section, IIT Kanpur.
%  %\item Pursued French as second language during secondary school from Grade 6 to Grade 10. Also participated in French Song Competition and French G.K. Quiz in Class 10th. %

% \end{rSection}

% \begin{rSection}{Personal Traits}

% \end{rSection}
\end{document}
